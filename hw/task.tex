\documentclass[10pt,a4paper,oneside]{article}
\usepackage[utf8]{inputenc}
\usepackage[english,russian]{babel}
\usepackage{amsmath}
\usepackage{amsthm}
\usepackage{amssymb}
\usepackage{enumerate}
\usepackage{stmaryrd}
\usepackage{cmll}
\usepackage{mathrsfs}
\usepackage[left=2cm,right=2cm,top=2cm,bottom=2cm,bindingoffset=0cm]{geometry}
\usepackage{proof}
\usepackage{tikz}
\usepackage{multicol}

\makeatletter
\newcommand{\dotminus}{\mathbin{\text{\@dotminus}}}

\newcommand{\@dotminus}{%
  \ooalign{\hidewidth\raise1ex\hbox{.}\hidewidth\cr$\m@th-$\cr}%
}
\makeatother

\usetikzlibrary{arrows,backgrounds,patterns,matrix,shapes,fit,calc,shadows,plotmarks}

\newtheorem{definition}{Определение}

\begin{document}

\begin{quote}
\selectlanguage{russian}

\begin{definition}
Рассмотрим следующие примитивы. 
\begin{enumerate}
\item $Z: \mathbb{N}_0 \rightarrow \mathbb{N}_0$, $Z(x) = 0$
\item $N: \mathbb{N}_0 \rightarrow \mathbb{N}_0$, $N(x) = x'$
\item Проекция. $U^n_i: \mathbb{N}_0^n \rightarrow \mathbb{N}_0$, $U^n_i (x_1, ... x_n) = x_i$
\item Подстановка. Если $f: \mathbb{N}_0^n \rightarrow \mathbb{N}_0$ и $g_1, ... g_n: \mathbb{N}_0^m \rightarrow \mathbb{N}_0$, 
  то $S\langle{}f,g_1,...g_n\rangle: \mathbb{N}_0^m \rightarrow \mathbb{N}_0$.
При этом $S\langle{}f,g_1,...g_n\rangle (x_1,...x_m) = f(g_1(x_1,...x_m), ... g_n(x_1,...x_m))$
\item Примитивная рекурсия. Если $f: \mathbb{N}_0^n \rightarrow \mathbb{N}_0$ и $g: \mathbb{N}_0^{n+2} \rightarrow \mathbb{N}_0$, то
  $R\langle{}f,g\rangle: \mathbb{N}_0^{n+1} \rightarrow \mathbb{N}_0$, при этом
  $$R\langle{}f,g\rangle (x_1,...x_n,y) = \left\{\begin{array}{ll}
    f(x_1,...x_n) & , y = 0\\
    g(x_1,...x_n,y-1,R\langle{}f,g\rangle(x_1,...x_n,y-1)) &, y > 0
  \end{array}\right.$$
\item Минимизация. Если $f: \mathbb{N}_0^{n+1} \rightarrow \mathbb{N}_0$, то $\mu \langle{}f\rangle: \mathbb{N}_0^n \rightarrow \mathbb{N}_0$, при этом
  $\mu \langle{}f\rangle (x_1,...x_n)$ --- такое минимальное число $y$, что $f(x_1,...x_n,y) \ne 0$.
  Если такого $y$ нет, результат данного примитива неопределен.
\end{enumerate}

Первые три из них --- обычные функции на натуральных числах. Оставшиеся три подобны шаблонам в C++ или 
функциям высшего порядка в Хаскеле/Окамле.
\end{definition}

Например, функция $f(x)=x+2$ может быть выражена через данные примитивы так: $f(x) = S\langle N,N\rangle (x)$.

\begin{definition}
Функция называется примитивно-рекурсивной, если возможно построить выражение только из первых пяти примитивов,
такое, что оно при всех аргументах возвращает значение, равное значению требуемой функции.

Если функция может быть выражена с помощью всех шести примитивов, она называется рекурсивной.
\end{definition}

Данное задание в целом сводится к демонстрации того, что различные функции являются
примитивно-рекурсивными (рекурсивными). В отличие от предыдущих заданий, в данном задании
это необходимо показывать при помощи демонстрации соответствующей программы на языке 
примитивно-рекурсивных (рекурсивных) функций.
Данный язык, например, легко эмулируется языком шаблонов C++ (как подсказывает синтаксис 
рекурсивных выражений), также возможно использовать любой другой интерпретатор.
\emph{Важно!} Функции должны работать --- демонстрация решения должна включать
запуск программы на тестах, предложенных преподавателем и/или товарищами.

\begin{enumerate}
\item Покажите, что следующие функции --- примитивно-рекурсивные:

\begin{enumerate}
\item сложение;
\item умножение;
\item ограниченное вычитание 1 (0 для 0, для остальных натуральных чисел совпадает с обычным вычитанием 1);
\item ограниченное вычитание (0, если $a < b$, и $a-b$, если $a \ge b$);
\item меньше: $m(a,b) = 1$, если $a < b$, иначе 0.
\item побитовая конъюнкция (операция \verb!&! в языке Си);
\item побитовое <<исключающее или>>;
\item конструкция $\textrm{first}\langle f \rangle(x_1,\dots,x_k,n)$: возвращает минимальный $t < n$, что
$f(x_1,\dots,x_k,t) \ne 0$, либо $n$, если функция равна 0 при всех $t \in 0\dots {n-1}$;
\item деление нацело (деление с округлением вниз);
\item остаток от деления нацело;
\item возведение в степень;
\item \emph{частичный логарифм} $\textrm{plog}_k(n)$ --- максимальное $p$, что $k^p$ делит $n$.
Например, $\textrm{plog}_6(72) = 2$;
\item факториал;
\item упорядоченную пару, т.е. набор из трёх функций (одно задание, на подпункты не делится):
\begin{enumerate}
\item левая проекция: $\pi_l (\langle a,b\rangle) = a$;
\item правая проекция: $\pi_r (\langle a,b\rangle) = a$;
\item построение пары: $\langle\rangle (a,b) = \langle a,b\rangle$;
\end{enumerate}
\item проверку числа на простоту;
\item простое число номер $k$.
\end{enumerate}

\item Будем называть гёделевой нумерацией списка следующую конструкцию. Пусть $\langle a_0, \dots, a_{n-1} \rangle$ --- некоторый
список натуральных чисел. Пусть $p_i$ --- это простое число номер $i$ (естественно, $p_0 = 2$).
Тогда гёделева нумерация этого списка $\ulcorner \langle a_0, a_1, \dots, a_{n-1} \rangle \urcorner = 
2^{a_0} \cdot 3^{a_1} \cdot \dots \cdot p_{n-1}^{a_{n-1}}$.
Например, $\ulcorner \langle 7,1,4 \rangle\urcorner = 2^7 \cdot 3^1 \cdot 5^4 = 240000$.

Покажите, что следующие функции являются примитивно-рекурсивными:

\begin{enumerate}
\item nil: гёделев номер пустого списка;
\item $\textrm{cons}(x,\ulcorner\langle a_0,\dots,a_{n-1}\rangle\urcorner) = \ulcorner\langle x,a_0,\dots,a_{n-1}\rangle\urcorner$;
\item head: функция, возвращающая голову списка;
\item tail: функция, возвращающая хвост списка;
\item получение элемента списка с номером $k$: $(\ulcorner\langle a_0, \dots, a_{n-1} \rangle\urcorner)_k = a_k$
\item len: длина списка;
\item (@): конкатенация списков;
\end{enumerate}

\item Назовём функцией Аккермана следующую функцию:

\begin{definition}Функцией \emph{Аккермана} мы назовем так определенную 
функцию:

$$A(m,n) = \left\{\begin{array}{rl}
   n+1, & \mbox{если $m = 0$}\\
   A(m-1,1), & \mbox{если $m > 0, n = 0$}\\
   A(m-1,A(m,n-1)), & \mbox{если $m > 0, n > 0$}
\end{array}\right.$$
\end{definition}

Покажите, что функция Аккермана --- рекурсивная (8 баллов).
К сожалению, примитивно-рекурсивной данная функция не является.
\end{enumerate}


\end{quote}

\end{document}

